\documentclass{article}
\usepackage[utf8]{inputenc}
\usepackage{amsmath}
\usepackage{amsfonts}
\newcommand{\Z}{\mathbb{Z}}
\newcommand{\N}{\mathbb{N}}
\newcommand{\Q}{\mathbb{Q}}
\newcommand{\R}{\mathbb{R}}

\title{Math}
\author{Martin Fosby }
\date{August 2018}

\begin{document}
\pagenumbering{gobble}

\paragraph{1.10}
\begin{align*}
   -5 \in \Z 
\end{align*}
\begin{align*}
   -5 \notin \N 
\end{align*}
\begin{align*}
   \frac{2}{3} \notin \Z
\end{align*}
\begin{align*}
  \frac{2}{3} \in \Q
\end{align*}
\begin{align*}
  \sqrt{5} \in \R
\end{align*}
\begin{align*}
  \sqrt{5} \notin \Q
\end{align*}

\paragraph{1.11}
\begin{align*}
  2 \in \{1,2,3,4\}
\end{align*}
\begin{align*}
  3 \notin \{0,1,2,4\}
\end{align*}
\begin{align*}
  1,5 \notin \{1,2,3,4\}
\end{align*}
\begin{align*}
  -1 \notin \{-2,1,0,1\}
\end{align*}

\paragraph{1.12}
\begin{align*}
  \{0,2,4,6,8,10,12,14,16,18,20\}
\end{align*}
\begin{align*}
  \{21,23,25,27,29,31,33,35\}
\end{align*}
\begin{align*}
  \{2,3,5,7,11,13,17,19,23,29\}
\end{align*}

\paragraph{1.13}
\begin{align*}
  \{1,2,3,4\} \backslash \{4\} = \{1,2,3\}
\end{align*}
\begin{align*}
  \{1,2,3,4\} \backslash \{2,4\} = \{1,3\}
\end{align*}
\begin{align*}
  \{1,2,3,4\} \backslash \{1,5\} = \{2,3,4\}
\end{align*}
\begin{align*}
  \Z \backslash \N = \{..., -4, -3, -2, -1, 0\}
\end{align*}

\paragraph{1.14}

\paragraph{a)}
\begin{align*}
  4 * 2^2 &= 4 * 4 \\
  &= 16
\end{align*}
\paragraph{b)}
\begin{align*}
  4 * (-2)^2 &= 4 * 4 \\
  &= 16
\end{align*}
\paragraph{c)}
\begin{align*}
  5 - 3^2 &= 5 - 9 \\
  &= -4
\end{align*}
\paragraph{d)}
\begin{align*}
  (5 - 3)^2 &= 2^2 \\
  &= 4
\end{align*}
\paragraph{e)}
\begin{align*}
  -2^2 + 3^2 -2 * (-2) &= -4 + 9 + 4 \\
  &= 9
\end{align*}
\paragraph{f)}
\begin{align*}
  -(-2)^2 + (-3)^2 - 2^2 &= -(4) + 9 - 4\\
  &= 1
\end{align*}
\paragraph{g)}
\begin{align*}
  (-3)^2 + 5 * (-3) + 6 &= 9 - 15 + 6 \\
  &= -5 + 6 \\
  &= 0
\end{align*}

\paragraph{Oppgave 1.15}

\paragraph{a)}
\begin{align*}
  2 (7 - 5) + 2 &= 2 * 2 + 2\\
  &= 4 + 2 = 6
\end{align*}

\paragraph{b)}
\begin{align*}
  -3(4-12)+2*3^2 &= -3 * -8 + 2 * 9\\
&= 24 + 18 = 42
\end{align*}

\paragraph{c)}
\begin{align*}
  -(8-4)-(3)^2 &= -4 - 9\\
&= -13
\end{align*}

\paragraph{d)}
\begin{align*}
  -2^4 + 3(17 - 3^2) + (3 * 4^2 - 2 * 5^2) &= -2^4 + 3 * 8 + (3*4^2-2*5^2)\\
  &= -2^4+3(17-3^2) - 2\\
  &=  -16 + 24 - 2\\
  &= 8 - 2\\
  &= 6
\end{align*}

\paragraph{Oppgave 1.16}
\paragraph{a)}
\begin{align*}
  2(2 * 2 - 2)^2 &= 2(4 -2)^2\\
&= 2 * 4\\
&= 8
\end{align*}

\paragraph{b)}
\begin{align*}
  -2^6 + (-2)^6 &= -64 + 64 = 0
\end{align*}
\paragraph{c)}
\begin{align*}
  4(3 - 2)^3 - 3(2 - 3)^3 &= 4 * 1 - 3 * (-1)\\
  &= 4 + 3\\
  &= 7
\end{align*}
\paragraph{d)}
\begin{align*}
  4(2^2 - 3)^5 - 3(2^3 - 3^2)^5 &= 4 * 1^5 - 3 * (-1)^5 \\
  &= 4 * 1 - 3 * -1 \\
  &= 4 - (-3) \\
  &= 4 + 3 \\
  &= 7
\end{align*}

\paragraph{Oppgave 1.17}
\paragraph{a)}
\begin{align*}
  2(2 * 2 - 2)^2 &= 2(4 -2)^2\\
&= 2 * 4\\
&= 8
\end{align*}

\paragraph{b)}
\begin{align*}
  -2^6 + (-2)^6 &= -64 + 64 = 0
\end{align*}
\paragraph{c)}
\begin{align*}
  4(3 - 2)^3 - 3(2 - 3)^3 &= 4 * 1 - 3 * (-1)\\
  &= 4 + 3\\
  &= 7
\end{align*}
\paragraph{d)}
\begin{align*}
  4(2^2 - 3)^5 - 3(2^3 - 3^2)^5 &= 4 * 1^5 - 3 * (-1)^5 \\
  &= 4 * 1 - 3 * -1 \\
  &= 4 - (-3) \\
  &= 4 + 3 \\
  &= 7
\end{align*}


\paragraph{exmaples}
\begin{align*}
  3 * \frac{2}{3} &= \frac{3}{1} * \frac{3}{3} - \frac{2}{3} \\
  &= \frac{7}{3} = \frac{3 + 3 + 1}{3} \\
  &= \frac{3}{3} + \frac{3}{3} + \frac{1}{3} \\
  &= 1 + 1 + \frac{1}{3} = 2\frac{1}{3} \\
\end{align*}

\paragraph{1.22}
\begin{align*}
  \frac{1}{12} + \frac{4}{9}
\end{align*}
\begin{align*}
  \frac{1}{12} * \frac{4}{9}
\end{align*}
\begin{align*}
  3 + \frac{5}{12}
\end{align*}
\begin{align*}
  3 * \frac{5}{12}
\end{align*}
\begin{align*}
  3 * \frac{5}{12}
\end{align*}

\paragraph{1.23}
\begin{align*}
  2 * (\frac{3}{8} + \frac{1}{4})
\end{align*}
\begin{align*}
  2 * (\frac{5}{6} + \frac{2}{9}) * \frac{3}{5}
\end{align*}
\begin{align*}
  (\frac{5}{36} + \frac{1}{12}) * \frac{2}{9}
\end{align*}
\begin{align*}
  (\frac{7}{6} + \frac{2}{9}) * (\frac{1}{5} + \frac{1}{4})
\end{align*}
\begin{align*}
\end{align*}


\paragraph{1.24}
\begin{align*}
  \frac{\frac{2}{3}}{\frac{5}{6}} = \frac{\frac{2}{3} * \frac{6}{1}}{\frac{5}{6} * \frac{6}{1}}
\end{align*}
\begin{align*}
  \frac{\frac{21}{36}}{\frac{14}{45}}
\end{align*}
\begin{align*}
  \frac{\frac{3}{2} + \frac{5}{8}}{\frac{1}{4} + \frac{25}{2}}
\end{align*}
\begin{align*}
  \frac{3 + \frac{4}{3}}{\frac{5}{12} + 5}
\end{align*}
\begin{align*}
\end{align*}

\end{document}
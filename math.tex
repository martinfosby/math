\documentclass{article}
\usepackage[utf8]{inputenc}
\usepackage{amsmath}
\usepackage{amsfonts}
\usepackage{cancel}
\newcommand{\Z}{\mathbb{Z}}
\newcommand{\N}{\mathbb{N}}
\newcommand{\Q}{\mathbb{Q}}
\newcommand{\R}{\mathbb{R}}

\title{Math}
\author{Martin Fosby }
\date{August 2018}

\begin{document}
\pagenumbering{gobble}

\paragraph{1.10}
\label{1.10}
\begin{align*}
   -5 \in \Z 
\end{align*}
\begin{align*}
   -5 \notin \N 
\end{align*}
\begin{align*}
   \frac{2}{3} \notin \Z
\end{align*}
\begin{align*}
  \frac{2}{3} \in \Q
\end{align*}
\begin{align*}
  \sqrt{5} \in \R
\end{align*}
\begin{align*}
  \sqrt{5} \notin \Q
\end{align*}

\paragraph{1.11}
\begin{align*}
  2 \in \{1,2,3,4\}
\end{align*}
\begin{align*}
  3 \notin \{0,1,2,4\}
\end{align*}
\begin{align*}
  1,5 \notin \{1,2,3,4\}
\end{align*}
\begin{align*}
  -1 \notin \{-2,1,0,1\}
\end{align*}

\paragraph{1.12}
\begin{align*}
  \{0,2,4,6,8,10,12,14,16,18,20\}
\end{align*}
\begin{align*}
  \{21,23,25,27,29,31,33,35\}
\end{align*}
\begin{align*}
  \{2,3,5,7,11,13,17,19,23,29\}
\end{align*}

\paragraph{1.13}
\begin{align*}
  \{1,2,3,4\} \backslash \{4\} = \{1,2,3\}
\end{align*}
\begin{align*}
  \{1,2,3,4\} \backslash \{2,4\} = \{1,3\}
\end{align*}
\begin{align*}
  \{1,2,3,4\} \backslash \{1,5\} = \{2,3,4\}
\end{align*}
\begin{align*}
  \Z \backslash \N = \{..., -4, -3, -2, -1, 0\}
\end{align*}

\paragraph{1.14}

\paragraph{a)}
\begin{align*}
  4 * 2^2 &= 4 * 4 \\
  &= 16
\end{align*}
\paragraph{b)}
\begin{align*}
  4 * (-2)^2 &= 4 * 4 \\
  &= 16
\end{align*}
\paragraph{c)}
\begin{align*}
  5 - 3^2 &= 5 - 9 \\
  &= -4
\end{align*}
\paragraph{d)}
\begin{align*}
  (5 - 3)^2 &= 2^2 \\
  &= 4
\end{align*}
\paragraph{e)}
\begin{align*}
  -2^2 + 3^2 -2 * (-2) &= -4 + 9 + 4 \\
  &= 9
\end{align*}
\paragraph{f)}
\begin{align*}
  -(-2)^2 + (-3)^2 - 2^2 &= -(4) + 9 - 4\\
  &= 1
\end{align*}
\paragraph{g)}
\begin{align*}
  (-3)^2 + 5 * (-3) + 6 &= 9 - 15 + 6 \\
  &= -5 + 6 \\
  &= 0
\end{align*}

\paragraph{Oppgave 1.15}

\paragraph{a)}
\begin{align*}
  2 (7 - 5) + 2 &= 2 * 2 + 2\\
  &= 4 + 2 = 6
\end{align*}

\paragraph{b)}
\begin{align*}
  -3(4-12)+2*3^2 &= -3 * -8 + 2 * 9\\
&= 24 + 18 = 42
\end{align*}

\paragraph{c)}
\begin{align*}
  -(8-4)-(3)^2 &= -4 - 9\\
&= -13
\end{align*}

\paragraph{d)}
\begin{align*}
  -2^4 + 3(17 - 3^2) + (3 * 4^2 - 2 * 5^2) &= -2^4 + 3 * 8 + (3*4^2-2*5^2)\\
  &= -2^4+3(17-3^2) - 2\\
  &=  -16 + 24 - 2\\
  &= 8 - 2\\
  &= 6
\end{align*}

\paragraph{Oppgave 1.16}
\paragraph{a)}
\begin{align*}
  2(2 * 2 - 2)^2 &= 2(4 -2)^2\\
&= 2 * 4\\
&= 8
\end{align*}

\paragraph{b)}
\begin{align*}
  -2^6 + (-2)^6 &= -64 + 64 = 0
\end{align*}
\paragraph{c)}
\begin{align*}
  4(3 - 2)^3 - 3(2 - 3)^3 &= 4 * 1 - 3 * (-1)\\
  &= 4 + 3\\
  &= 7
\end{align*}
\paragraph{d)}
\begin{align*}
  4(2^2 - 3)^5 - 3(2^3 - 3^2)^5 &= 4 * 1^5 - 3 * (-1)^5 \\
  &= 4 * 1 - 3 * -1 \\
  &= 4 - (-3) \\
  &= 4 + 3 \\
  &= 7
\end{align*}

\paragraph{Oppgave 1.17}
\paragraph{a)}
\begin{align*}
  2(2 * 2 - 2)^2 &= 2(4 -2)^2\\
&= 2 * 4\\
&= 8
\end{align*}

\paragraph{b)}
\begin{align*}
  -2^6 + (-2)^6 &= -64 + 64 = 0
\end{align*}
\paragraph{c)}
\begin{align*}
  4(3 - 2)^3 - 3(2 - 3)^3 &= 4 * 1 - 3 * (-1)\\
  &= 4 + 3\\
  &= 7
\end{align*}
\paragraph{d)}
\begin{align*}
  4(2^2 - 3)^5 - 3(2^3 - 3^2)^5 &= 4 * 1^5 - 3 * (-1)^5 \\
  &= 4 * 1 - 3 * -1 \\
  &= 4 - (-3) \\
  &= 4 + 3 \\
  &= 7
\end{align*}

\paragraph{Oppgave 1.20}
\paragraph{a)}
\begin{align*}
  \frac{4}{6} &= \frac{4:2}{6:2} \\
  &= \frac{2}{3}
\end{align*}

\paragraph{b)}
\begin{align*}
  \frac{9}{15} &= \frac{9:3}{15:3} \\
  &= \frac{3}{5}
\end{align*}
\paragraph{c)}
\begin{align*}
  \frac{18}{21} &= \frac{18:3}{21:3} \\
  &= \frac{6}{7}
\end{align*}
\paragraph{d)}
\begin{align*}
  \frac{42}{54} &= \frac{42:6}{54:6} \\
  &= \frac{7}{9}
\end{align*}

\paragraph{Oppgave 1.21}
\paragraph{a)}
\begin{align*}
  \frac{72}{120} &= \frac{72:8}{120:8} \\
  &= \frac{9}{15} \\
  &= \frac{9:3}{15:3} \\
  &= \frac{3}{5} \\
\end{align*}

\paragraph{b)}
\begin{align*}
  \frac{126}{294} &= \frac{126:7}{294:7} \\
  &= \frac{18:2}{42:2} \\
  &= \frac{9:3}{21:3} \\
  &= \frac{3}{7} \\
\end{align*}
\paragraph{c)}
\begin{align*}
  \frac{132}{198} &= \frac{132:2}{198:2} \\
  &= \frac{66:3}{99:3} \\
  &= \frac{22:11}{33:11} \\
  &= \frac{2}{3}
\end{align*}
\paragraph{d)}
\begin{align*}
  \frac{153}{51} &= \frac{153:3}{51:3} \\
  &= \frac{51}{17}
\end{align*}


\paragraph{exmaples}
\begin{align*}
  3 * \frac{2}{3} &= \frac{3}{1} * \frac{3}{3} - \frac{2}{3} \\
  &= \frac{7}{3} = \frac{3 + 3 + 1}{3} \\
  &= \frac{3}{3} + \frac{3}{3} + \frac{1}{3} \\
  &= 1 + 1 + \frac{1}{3} = 2\frac{1}{3} \\
\end{align*}

\paragraph{1.22}
\begin{align*}
  \frac{1}{12} + \frac{4}{9} &= \frac{9}{9} * \frac{1}{12} + \frac{4}{9} * \frac{12}{12} \\
  &= \frac{9}{108} + \frac{48}{108} \\
  &= \frac{57:3}{108:3} \\
  &= \frac{19}{36}
\end{align*}
\begin{align*}
  \frac{1}{12} * \frac{4}{9} &= \frac{4:2}{108:2} \\
  &= \frac{2:2}{54:2} \\
  &= \frac{1}{27}
\end{align*}
\begin{align*}
  \frac{1}{12} : \frac{4}{9} &= \frac{1}{12} * \frac{9}{4} \\
  &= \frac{9:3}{48:3} \\
  &= \frac{3}{16}
\end{align*}
\begin{align*}
  3 + \frac{5}{12} &= \frac{3 * 12}{1 * 12} + \frac{5}{12} \\
  &= \frac{36}{12} + \frac{5}{12} \\
  &= \frac{41}{12}
\end{align*}
\begin{align*}
  3 * \frac{5}{12} &= \frac{3}{1} * \frac{5}{12} \\
  &= \frac{15:3}{12:3} \\
  &= \frac{5}{4}
\end{align*}
\begin{align*}
  3 : \frac{5}{12} &= \frac{3}{1} : \frac{5}{12} \\
  &= \frac{3}{1} * \frac{12}{5} \\
  &= \frac{36}{5}
\end{align*}

\paragraph{1.23}
\begin{align*}
  2 * \left(\frac{3}{8} + \frac{1}{4}\right) &= 2 * \left(\frac{4}{4} * \frac{3}{8} + \frac{1}{4} * \frac{8}{8}\right) \\
  &= 2 * \left(\frac{12}{32} + \frac{8}{32}\right) \\
  &= 2 * \frac{20}{32} \\
  &= \frac{2 * 20}{32} \\
  &= \frac{40:2}{32:2} \\ 
  &= \frac{20:2}{16:2} \\
  &= \frac{10:2}{8:2} \\
  &= \frac{5}{4}
\end{align*}
\begin{align*}
  \left(\frac{5}{6} - \frac{2}{9}\right) * \frac{3}{5} &= \left(\frac{5*3}{6*3} - \frac{2*2}{9*2}\right) * \frac{3}{5} \\
  &= \left(\frac{15}{18} - \frac{4}{18}\right) * \frac{3}{5} \\
  &= \frac{11}{18} * \frac{3}{5} \\
  &= \frac{33:3}{90:3} \\ 
  &= \frac{11}{30}
\end{align*}
\begin{align*}
  \left(\frac{5}{36} + \frac{1}{12}\right) : \frac{2}{9} &= \left(\frac{5}{36} + \frac{1*3}{12*3}\right) : \frac{2}{9} \\
  &= \left(\frac{5}{36} + \frac{3}{36}\right) : \frac{2}{9} \\
  &= \frac{8}{36} * \frac{9}{2} \\
  &= \frac{72}{72} \\
  &= 1
\end{align*}
\begin{align*}
  \left(\frac{7}{6} - \frac{2}{9}\right) * \left(\frac{1}{5} + \frac{1}{4}\right) &= \left(\frac{7*3}{6*3} - \frac{2*2}{9*2}\right) * \left(\frac{1*4}{5*4} + \frac{1*5}{4*5}\right) \\
  &= \left(\frac{21}{18} - \frac{4}{18}\right) * \left(\frac{4}{20} + \frac{5}{20}\right) \\
  &= \frac{17}{18:2} * \frac{9}{20} \\
  &= 
  &= \frac{17}{40} \\ 

\end{align*}

\paragraph{1.24}
\begin{align*}
  \frac{\frac{2}{3}}{\frac{5}{6}} = \frac{\frac{2}{3} * \frac{6}{1}}{\frac{5}{6} * \frac{6}{1}}
\end{align*}
\begin{align*}
  \frac{\frac{21}{36}}{\frac{14}{45}}
\end{align*}
\begin{align*}
  \frac{\frac{3}{2} + \frac{5}{8}}{\frac{1}{4} + \frac{25}{2}}
\end{align*}
\begin{align*}
  \frac{3 + \frac{4}{3}}{\frac{5}{12} + 5}
\end{align*}



\paragraph{1.30}
\begin{align*}
  2x - 5y + 3x + 7y + 1 &= 5x + 2y + 1 
\end{align*}
\begin{align*}
  a^2 + 2a + 3 + a^2 - 3a -1 &= 2a^2 - a + 2
\end{align*}
\begin{align*}
  2x^2 + x + y^2 - 2x - 2y^2 &= 2x^2 - x - y^2
\end{align*}
\begin{align*}
  2xy + xy^2 - x^2y - 2xy^2 - yx &= xy - xy^2 - x^2y
\end{align*}

\paragraph{1.31}
\begin{align*}
  (5x + y) + (2x - y) &= 5x + y + 2x - y \\
  &= 7x 
\end{align*}
\begin{align*}
  a + 2b - (-a + b) &= a + 2b + a - b \\
  &= 2a + b
\end{align*}
\begin{align*}
  (x^2 + 2x + 1) - (x^2 - 2x + 1) &= x^2 + 2x + 1 - x^2 + 2x - 1 \\
  &= 4x 1 - 1 \\
  &= 4x
\end{align*}
\begin{align*}
  2a^2 - a - 3 + (-a^2 + a + 3) &= 2a^2 - a - 3 - a^2 + a + 3 \\
  &= a^2 
\end{align*}

\paragraph{1.32}
\begin{align*}
  2(x + 4) &= 2*x + 2 * 4 \\
  &= 2x + 8
\end{align*}
\begin{align*}
  -2(t - 3) &= -2 * t - (-2 * 3) \\
  &= -2t + 6
\end{align*}
\begin{align*}
  3(2x + 1 ) - 2(3x + 1) &= 3 * 2x + 3 * 1 - 2 * 3x - 2 * 1 \\
  &= 6x + 3 - 6x - 2 \\
  &= 3 - 2 \\ 
  &= 1
\end{align*}
\begin{align*}
  5(x^2 + 3x + 2) - 5(x^2 + 1) &= 5x^2 + 15x + 10 - 5x^2 - 5 \\
  &= 15x + 10 - 5 \\
  &= 15x + 5
\end{align*}

\paragraph{1.33}
\begin{align*}
  2(2a - b) + 3(-2a + 3b) &= 4a - 2b - 6a + 9b \\
  &= -2a + 7b
\end{align*}
\begin{align*}
  2a(ab - b^2) - 2b(a^2 - ab) &= 2ba^2 - 2ab^2 - 2ba^2 + 2ab^2 \\
  &= 2ba^2 - 2ba^2 + 2ab^2 - 2ab^2 \\
  &= 0
\end{align*}
\begin{align*}
  (x + 1)(2x - 3) &= 
\end{align*}
\begin{align*}
  5(x^2 + 3x + 2) - 5(x^2 + 1) &= 5x^2 + 15x + 10 - 5x^2 - 5 \\
  &= 15x + 10 - 5 \\
  &= 15x + 5
\end{align*}

\paragraph{1.40}
\begin{align*}
  \frac{a}{2} + \frac{a}{3} + \frac{a}{6} &= \frac{a}{2*3} + \frac{a}{3*2} + \frac{a}{6} \\
  &= \frac{a}{6} + \frac{a}{6} + \frac{a}{6} \\
  &= \frac{x}{y}
\end{align*}
\begin{align*}
  2a(ab - b^2) - 2b(a^2 - ab) &= 2ba^2 - 2ab^2 - 2ba^2 + 2ab^2 \\
  &= 2ba^2 - 2ba^2 + 2ab^2 - 2ab^2 \\
  &= 0
\end{align*}
\begin{align*}
  (x + 1)(2x - 3) &= 
\end{align*}
\begin{align*}
  5(x^2 + 3x + 2) - 5(x^2 + 1) &= 5x^2 + 15x + 10 - 5x^2 - 5 \\
  &= 15x + 10 - 5 \\
  &= 15x + 5
\end{align*}
\end{document}